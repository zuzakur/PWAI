\documentclass{article}
\usepackage[utf8]{inputenc}
\usepackage[polish]{babel}
\usepackage[T1]{fontenc} 
\usepackage{listings, amssymb, amsmath, geometry, hyperref, verbatim, float, tikz, setspace, fancybox}

\definecolor{darkgreen}{rgb}{0, 0.5, 0}
\definecolor{darkblue}{rgb}{0, 0, 0.5}
\lstset{
	basicstyle={\ttfamily\small}, frame=single, language=Python, framexleftmargin=0pt,
	commentstyle=\color{darkgray}\textit, keywordstyle=\color{darkblue}\bfseries, stringstyle=\color{darkgreen}, showstringspaces=false,
     literate={ą}{{\k a}}1
	{Ą}{{\k A}}1
	{ż}{{\. z}}1
	{Ż}{{\. Z}}1
	{ź}{{\' z}}1
	{Ź}{{\' Z}}1
	{ć}{{\' c}}1
	{Ć}{{\' C}}1
	{ę}{{\k e}}1
	{Ę}{{\k E}}1
	{ó}{{\' o}}1
	{Ó}{{\' O}}1
	{ń}{{\' n}}1
	{Ń}{{\' N}}1
	{ś}{{\' s}}1
	{Ś}{{\' S}}1
	{ł}{{\l}}1
	{Ł}{{\L}}1
    }
\geometry{margin=3cm}


\title{Projekt zaliczeniowy (Raport) - Podstawowy Warsztat AI}
\author{Zuzanna Kurek i Szymon Gajdziszewski}
\date{Styczeń 2026}
\begin{document}
\maketitle
\tableofcontents

\section{Wstęp}
Projekt obejmuje pobranie i skonsolidowanie historycznych danych pogodowych udostępnianych przez Instytut Meteorologii i Gospodarki Wodnej (IMGW) oraz wykonanie prostej analizy wybranych danych. W poniższym raporcie zostały opisane 

\begin{itemize}
    \item wykonane kroki;
    \item fragmenty użytego kodu;
    \item uzyskane w konsekwencji wyniki;
    \item oraz wyciągnięte wnioski.
\end{itemize}

\newpage

\section{Konsolidacja danych}
\subsection{Pobieranie}

Pliki do późniejszej analizy należało pobrać ze strony o adresie \url{https://danepubliczne.imgw.pl/data/dane_pomiarowo_obserwacyjne/dane_meteorologiczne/dobowe/klimat/}. Do tego celu napisano krótki skrypt w Pythonie. Zaczęto od stworzenia docelowego folderu:

\begin{lstlisting}
OUTPUT_DIR = "dane_imgw"
os.makedirs(OUTPUT_DIR, exist_ok=True)
\end{lstlisting}

Następnie, za pomocą pętli \texttt{for}, biblioteki \texttt{urllib.request} i \texttt{zipfile}, pobrano osobno pliki zip z każdego miesiąca każdego roku i rozpakowano je do \verb|dane_imgw|.

\subsection{Scalanie}

Napisano skrypt w Pythonie, w którym zaczęto od stworzenia pustej listy o odpowiednim rozmiarze:

\begin{lstlisting}
full_data = np.empty((0, 18), dtype=object)
\end{lstlisting}

Następnie pętlą \texttt{for} wpisano do niej odpowiednie dane z wcześniej uzyskanych plików:

\begin{lstlisting}
for year in range(2001, 2024):
    for month in range(1, 13):
        filename = f"k_d_{month:02d}_{year}.csv"
        data = np.loadtxt(filename, delimiter=',', dtype=str,
        encoding="cp1250", quotechar='"')
        full_data = np.vstack((full_data, data))
\end{lstlisting}

Usunięto zbędnę kolumny przy użyciu \texttt{np.delete}, usunięto cudzysłowia i zapisano do nowego pliku \texttt{.csv}.

\newpage

\section{Proste charakterystyki i statystyki danych}

\subsection{Stacje z pełną historią danych}

Kluczowym krokiem było odpowiednie wczytanie danych w następujący sposób:

\begin{lstlisting}
station_ids = np.genfromtxt(filename, dtype=str, delimiter=None,
usecols = 0, encoding="cp1250", filling_values="")
\end{lstlisting}

\texttt{usecols = 0} pozwala na wczytanie jedynie pierwszej kolumny (a więc tej zawierającej numer stacji. Następnie zliczono ilość występowania każdego numeru i sprawdzeniu czy jest równa \texttt{8400} (ponieważ tyle wystąpień powinna mieć stacja, która posiada pełną historię). Do tego użyte zostało \texttt{np.unique} i maski:

\begin{lstlisting}
ID, counts = np.unique(station_ids, return_counts=True)

full_days = 8400
mask_full_history = counts == full_days
stacje_full = ID[mask_full_history]
stacje_nfull = ID[~mask_full_history]
\end{lstlisting}

\newpage

\subsection{Prezentacja graficzna ilości pomairów wybranej stacji}

Za pomocą funkcji \texttt{np.genfromtxt} wczytano kolumny odpowiadające za numer ID stacji, rok oraz miesiąc. Stosując maskę logiczną oddzielono dane pasujące tylko do analizowanej stacji. Zagnieżdżoną pętlą dla każdego miesiąca w każdym roku zaliczono liczbę wystąpień pomariarów. Wynik zapisywany jest w odpowiedniej komórce wcześniej zrobionej macierzy $23 \times 12$. 

\begin{lstlisting}
for i, rok in enumerate(years):
for j, miesiac in enumerate(months):
    warunek = (dane_stacji[:, 1] == rok) & (dane_stacji[:, 2] == miesiac)
    macierz_pomiarów[i, j] = np.sum(warunek)
\end{lstlisting}

\begin{figure}[H]
    \centering
    \includegraphics[width=1\textwidth]{wykres10.png}
    \caption{Wykres ilości pomiarów stacji 249190480}
    \label{fig:Wykres ilości pomiarów stacji 249190480}
\end{figure}

Do sporządzenia wykresu stacji o ID 249190480 użyto 

\begin{lstlisting}
plt.colorbar(label='Liczba pomiarów w miesiącu')
plt.xticks(np.arange(len(months)), months)
plt.yticks(np.arange(len(years)), years)
\end{lstlisting}

\subsection{Wykres średniej dziennej temperatury w zależności
od liczby dni, jakie upłynęły od 01.01.2001 (wersja dokładna)}

Po wczytaniu danych za pomocą \texttt{np.genfromtxt} stworzono maski:

\begin{lstlisting}
maska_stacji = data_subset[:, 0] ==int(wybrana_stacja)
dane_stacji = data_subset[maska_stacji]
\end{lstlisting}

\begin{figure}[H]
    \centering
    \includegraphics[width=1\textwidth]{wykres_s.png}
    \caption{Wykres średniej dziennej temperatury w zależności
od liczby dni, jakie upłynęły od 01.01.2001}
    \label{fig:wykres}
\end{figure}

\subsection{Minimalna i maksymalna temperatura}

Wybrano kolumny odpowiadające za minimalną i maksymalną temperaturę, następnie wyszukano odpowiednio najmniejszych i największych wartości. Maski umożliwiły na znalezienie wiersza, w którym znajduje się wartość i z tych wierszy odczytano potrzebne dane (ID stacji, nazwę stacji, dzień, miesiąc, rok) z odpowiednich kolumn.
Użycie masek:

\begin{lstlisting}
maska_min = (t_min_values == min_temp)
maska_max = (t_max_values == max_temp)
rekordy_min = data_subset[maska_min]
rekordy_max = data_subset[maska_max]
\end{lstlisting}

Wynik:

\begin{lstlisting}
Stacja: 249190560, JABŁONKA
Data: 8/1/2017
Stacja: 251150060,CEBER
Data: 08/08/2015
\end{lstlisting}

\newpage

\subsection{Różnica dziennych średnich temperatur dwóch stacji}

W tym podpunkcie zastosowane zostały maski, pętla \texttt{for} oraz funkcje numpy \texttt{np.argpartition()} i \texttt{np.argsort} do znalezienia indeksów odpowiadającym dniom z największą różnicą średnich temperatur.

\begin{lstlisting}
top_idx = np.argpartition(s_roznica_temp, -5)[-5:]
top_idx = top_idx[np.argsort(s_roznica_temp[top_idx])[::-1]]
\end{lstlisting}

\begin{figure}[H]
    \centering
    \includegraphics[width=1\textwidth]{wykres5a.png}
    \caption{Wykres różnicy średnich temperatur dla dwóch stacji z pełnymi historiami}
    \label{fig:wykres5a}
\end{figure}

Skrypt zwraca też odpowiedź na pytanie "W jakich dniach różnice są
największe?", podając pięć największych wartości:

\begin{itemize}
\item 5.90 stopni – w dniu 2001 - 1 - 10
\item 5.90 stopni – w dniu 2001 - 1 - 15
\item 5.10 stopni – w dniu 2001 - 1 - 11
\item 5.10 stopni – w dniu 2001 - 1 - 16
\item 4.50 stopni – w dniu 2001 - 2 - 16
\end{itemize}

\newpage

\subsection{Wykywanie anomalii}

Anomalię zdefiniowano jako obserwację, dla której wartość temperatury odbiega od średniej temperatury dla danej stacji o więcej niż trzykrotność odchylenia standardowego. Dla stacji o ID 249190560 obliczono średnią temperaturę oraz odchylenie standardowe, a następnie wyznaczono anomalie. 

\begin{lstlisting}
srednia = np.mean(temperatury)
sigma = np.std(temperatury)
prog = 3
maska_anomalii = np.abs(temperatury - srednia) > prog * sigma
anomalie = dane_stacji[maska_anomalii]
\end{lstlisting}

Wykryto 31 anomalii, z których wypisano 5:

\begin{itemize}
    \item 04.01.2002 – -24.9°C
    \item25.12.2002 – -24.6°C
    \item 09.01.2003 – -19.7°C
    \item 13.02.2003 – -20.0°C
    \item 25.12.2003 – -19.7°C
\end{itemize}

oraz sporządzono wykres:

\begin{figure}[H]
    \centering
    \includegraphics[width=1\textwidth]{anomalie.png}
    \caption{Wykres anomalii}
    \label{fig:wykres3.6}
\end{figure}

\newpage

\section{Wnioski}

\subsection {Stacje z pełną historią danych}

Znacznie więcej wykryto stacji z niepełną historią co wskazuje na występowanie wydarzeń typu awarie, sugeruje, że niektóre stacje mogły zostać zamknięte lub otwarte w okresie, który był analizowany

\subsection{Prezentacja graficzna ilości pomairów wybranej stacji}

Dla wybranej stacji można dostrzec, że mniej pomiarów wykonywanych było w miesiące o mniejszej ilości dni (lipiec 2019). Widać też pojedyncze miesiące o mniejszej ilości wykonanych pomiarów (luty), w których najpewniej doszło do awarii lub wykonywano prace konserwacyjne.

\subsection{Wykres średniej dziennej temperatury w zależności
od liczby dni, jakie upłynęły od 01.01.2001 (wersja dokładna)}

Zgodnie z podejrzeniami średnie temperatury były co roku najniższe w miesiącach zimowych, a najwyższe w letnich. Zaskakujące jest występowanie aż trzech dni w przeciągu analizowanych lat, w których średnia temperatura wynosiła -20 stopni.

\subsection{Minimalna i maksymalna temperatura}

Z uzyskanych danych można zgodnie z przewidywaniami wywnioskować, że najniższa zarejestrowana temperatura panowała w Jabłonce (miejscowość na południu Małopolski) w styczniu, a najwyższa w Ceber (Dolny Śląsk - potencjalnie najcieplejszy obszar w Polsce) w sierpniu.

\subsection{Różnica dziennych średnich temperatur dwóch stacji}

Z wykresu można wywnioskować, że różnica średnich zmierzonych temperatur pomiędzy wybranymi stacjami zawiera się w przedziale od -9 do 6 stopni. Są to istotne różnice, co wstazuje na to prawdopodobny duży dystans pomiędzy dwiema stacjami. Wielkość różnic zmieniała się rocznie z pewną regularnością.

\subsection{Wykywanie anomalii}

Większość pomiarów temperatury mieści się w typowym zakresie wahań wokół średniej, natomiast wykryte anomalie odpowiadają pojedynczym, silnym odchyleniom. Anomalie występują głównie w okresach zimowych i mają charakter sporadyczny.

\end{document}
